\documentclass[11pt]{article}

% \usepackage[sort]{natbib}
\usepackage[style=verbose]{biblatex}
\usepackage{fancyhdr}
\usepackage{graphicx,caption,subcaption,color,float} %Graphics stuff
\usepackage{hyperref,amssymb,amsmath, amsfonts, amsthm, enumerate, bm}
\usepackage{placeins, cancel, wrapfig, xcolor, array, multirow, booktabs, algorithm, algpseudocode} 
\usepackage[margin=0.9in]{geometry}
\usepackage{ulem}
\graphicspath{ {figs/} }
\bibliography{references}

% you may include other packages here (next line)
\usepackage{enumitem}
\usepackage{dirtytalk}

%----- you must not change this -----------------
\topmargin -1.0cm
\textheight 23.0cm
\parindent=0pt
\parskip 1ex
\renewcommand{\baselinestretch}{1.1}
\pagestyle{fancy}
\renewcommand{\theenumi}{\Alph{enumi}}
\makeatletter
\newcommand{\distas}[1]{\mathbin{\overset{#1}{\kern\z@\sim}}}%
\newsavebox{\mybox}\newsavebox{\mysim}
\newcommand{\distras}[1]{%
  \savebox{\mybox}{\hbox{\kern3pt$\scriptstyle#1$\kern3pt}}%
  \savebox{\mysim}{\hbox{$\sim$}}%
  \mathbin{\overset{#1}{\kern\z@\resizebox{\wd\mybox}{\ht\mysim}{$\sim$}}}%
}
\makeatother
%----------------------------------------------------

% enter your details here----------------------------------
\lhead{}
\chead{}
\rhead{}
\lfoot{}
\cfoot{}
\rfoot{}
\setlength{\fboxrule}{4pt}\setlength{\fboxsep}{2ex}
\renewcommand{\headrulewidth}{0.4pt}
\renewcommand{\footrulewidth}{0.4pt}


\title{Homework 1}
\author{Your Name: Dong Ding}

\begin{document}

\maketitle
\textbf{Problem 1:}
\text{We have to make sure the matrix $I-\alpha A$ is invertible,} \text{otherwise the linear system has no solution.  } 
\[det(I-\alpha A)=0\]
\text{is equivalent to }
\[det(A-I/\alpha )=0\]
\text{We see that the largest value of $1/\alpha$}
\text{for which the determinant}
\text{is zero is the}
\text{the largest eigenvalue of matrix A.}
\text{In practice}
\text{$\alpha$ is often set close to the threshold 1/k1.}

\clearpage


\textbf{Problem 2:}
\text{The number of walks of length 1 between Vi and Vj} 
\text{is denoted by Aij} \par
\text{Number of walks of size 2 from Vi to Vj that go through Vk}
\text{is equivalent to }
\[ N_{ij}^{(2)} = \sum_{k=1}^{n} A_{ik} A_{kj} = [A^{2}]_{ij} \]




\clearpage

\textbf{Problem 3:}
\text {The idea is to define a function which calculate the Jaccard matrix by the definition,}\par 
\text {which is the common neighbors devided by their union neighbors.}\par
\text {Thus, we need to convert the graph into its}
\text{adjacency matrix. From question 2, we can get the common }\par
\text{neighbors by multiplying the adjacency matrix.}\par
\text{The union is their total neighbors minus the intersection. The code is shown below.}
\begin{figure}[h]
\includegraphics[width=8cm]{"pic1.png"}
\centering
\end{figure}

\begin{figure}[h]
\includegraphics[width=8cm]{"pic2.png"}
\centering
\end{figure}

\clearpage

\begin{figure}[h]
\includegraphics[width=8cm]{"Figure_1.png"}
\centering
\end{figure}


\text{The code to access the Ginori family}

\begin{figure}[h]
\includegraphics[width=8cm]{"pic3.png"}
\centering
\end{figure}

\begin{figure}[h]
\includegraphics[width=8cm]{"Figure_2.png"}
\centering
\end{figure}
\clearpage



\end{document}